\subsection{编译型语言与解释型语言}


\begin{frame}{编译型语言}

    编译型语言要求使用编译器一次性将所有源代码编译为一个可执行程序,一次编译可重复执行。

    \begin{myoutline}
        \1 编译型语言一般不能跨平台
            \2 编译出来的可执行程序不能跨平台:因为不同操作系统对可执行文件有着不同的要求,彼此之间不能兼容。
            \2 源代码不能跨平台: 不同操作系统下的函数、变量、api等可能会有不同。
        \1 代表语言:
            \2 C、C++
            \2 Golang
            \2 Rust
            \2 \textcolor{gray}{Java}?
    \end{myoutline}

\end{frame}
% Java即是编译型的,也是解释型语言
\begin{frame}{解释型语言}
    解释型语言是使用解释器一边执行一边转换,用到些源代码就转换哪些,不会生成可执行程序。
    \begin{myoutline}
        \1 解释型语言一般可以跨平台
            \2 跨平台是指源代码可以跨平台,解释器是不能跨平台的
            \2 源代码在不同操作系统中运行的结果相同
            \2 一个语言可以有不同的解释器用不同的语言实现这个解释器
        \1 代表语言:
            \2 C\#
            \2 JavaScript
            \2 PHP
            \2 Ruby
            \2 R
            \2 Python
    \end{myoutline}
\end{frame}
% 将高级语言的一条语句翻泽为机器语言,然后运行。且解释器发现错误, 程序会抛出异常或立即终止。

\begin{frame}{编译器与解释器}
    \begin{myoutline}
        \1 编译器 (编译型语言)
            \2 编译器在编译的过程中,读入源程序文件,输出一份等价的二进制可执行文件,就和笔译工作者一样,他们都会输出一份翻译后的文件。
        \1 解释器 (解释型语言)
            \2 解释器在解释的过程中,读入源程序文件,输出的是执行的结果,就和口译工作者一样,他们输出的是已经完成翻译的结果。
    \end{myoutline}
    \bigskip
    \bigskip
    \bigskip
    \centering

    \small{输出的不同是这两者最大的区别,一个会\textcolor{red}{输出用于执行的文件},另一个只会\textcolor{red}{输出运行的结果}。}
    \footnotenoindex{https://zhuanlan.zhihu.com/p/389371438}
\end{frame}


\begin{frame}{Python是一种解释型语言(了解)}
    \tiny{\textcolor{gray}{超纲内容, 入门同学可以先听不懂!}}
    \begin{myoutline}
        \1 导入大型包的时候第一次导入很慢, 以后会很快的原因?
        \1 Python并非完全是解释性语言, 它是\textcolor{red}{有编译}的:
            \2 Python在解释源程序时是分成两个步骤的:
                \3 首先处理py中的源代码, 编译生成一个二进制pyc\textcolor{red}{字节码文件} (main + module)
                \3 再对字节码进行处理, 才会生成CPU能够识别的机器码
                \3 有了module的字节码文件之后, 下一次运行程序时, 如果在上次保存字节码之后没有修改过源代码, Python 将会加载.pyc文件并\textcolor{red}{跳过编译module的字节码这个步骤}
                \3 当Python重编译时, 它会自动检查module源文件和字节码文件的时间戳
                \3 如果你又修改了module源代码,下次程序运行时,module的字节码pyc文件将自动重新创建
            \2 相对于py文件来说, 编译成pyc本质上和py没有太大区别, 只是对于\textcolor{red}{这个模块}的加载速度提高了;并没有提高代码的执行速度
            \2 通常情况下不用主动去编译pyc文件,除非需要隐藏源代码保持私密性
        \1 ``.pyc''文件可以对module导入进行加速, Python CLI 程序设计原则:
            \2 在 import 别的 py 文件(module, 模块)时,那个 py 文件会被存一份 pyc 加速下次装载
            \2 而主文件因为是直接执行而不是import导入所以不会保留pyc
            \2 这也是为什么写大程序原则是CLI入口代码尽量少的原因:
                \3 将主要代码都构建在模块中以加速程序启动
                \3 \textcolor{red}{因为CLI入口是当前运行的主程序, 并不会生成pyc!}
        \1 演示
    \end{myoutline}
\end{frame}

% (base)  Downloads  cat main.py
% from my_module import myprint

% myprint()
% (base)  Downloads  cat my_module.py
% def myprint():
%     print("a function in module!")

% 讲到这里,我们关于编译型语言和解释型语言的一些概念阐述就都说完了,
% 我们讲完系统,讲完编程语言的类别,那么下一步,我们要开始讲 Python 相关的内容了, 
% 其中会有很多实操课, 再进行下面的课程之前呢,希望能给大家提几个要求

% https://www.bilibili.com/video/BV1ex411x7Em?p=258&vd_source=e5c433ab0a245566be70de4210cd0727

\begin{frame}[standout]{接下来的要求}
    \begin{myoutline}
        \1 听课要求(零基础)
            \2 手机 No!
            \2 iPad No!
            \2 电脑 Yes!
        \1 养成记笔记的习惯
            \2 使用 Markdown 来记笔记
            \2 使用一个开箱即用的笔记工具
        \1 心态要求
            \2 看别人码代码是很枯燥的
            \2 动手自己跟着码代码, 获得成就感
            \2 程序员/生信工作者 90\%的时间是在解决问题的路上以及学习, 只有 10\%的时间在流畅地写代码, 要从Debug 的过程中得到进步和获得快乐!
        \1 不懂的概念和知识点
            \2 读报错信息, 动脑思考, 如果还不行↓
            \2 百度, 如果还不行↓
            \2 Google, 如果还不行↓
            \2 群里问, 请教师兄师姐
    \end{myoutline}
\end{frame}

% 希望大家, 尤其是零基础的同学, 看视频的时候能够边看边做, 手机和 iPad 学习编程都是非常低效的, 请拿起你的电脑
% 好记性不如烂笔头,多记心得体会和知识点, 至少以后有个印象回来翻翻, 记笔记的时候推荐使用支持 Markdown 的工具,比如某道云笔记,某象笔记,尽量开箱即用不花时间研究笔记工具
% 后面会有大量的代码课程, 其实看别人写代码是很容易走神儿的,所以希望你能自己动手敲一遍
% 最后一个点就是,遇到问题不要直接就问, 先思考,看报错信息,看不懂先把报错信息粘贴到百度或者谷歌上看看别人如何解决的
% 如果都没有解决你的问题, 来群里问

% 好的, 到这里我们就讲完了一些基本的要求





