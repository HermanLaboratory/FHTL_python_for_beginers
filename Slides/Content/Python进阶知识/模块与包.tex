\subsection{模块与包}
\begin{frame}[standout] 模块与包 \end{frame}
\begin{frame}{模块与包的概念}
    \begin{columns}
        \column{0.5\textwidth}
        \begin{myoutline}
            \1 内建对象(Built-in Bbjects)
                \2 print(), str()
                \2 int, str, float
                \2 dir(\_\_builtins\_\_) 
            \1 模块(Module)
                \2 需要import导入
                \2 使用 Python 编写的模块.py
                \2 使用 C 编写的动态加载模块(.dll, .pyd, .so, .sl等)
                \2 内建的 C 编写的模块 sys.builtin\_module\_names
            \1 包(Package)
                \2 需要import导入
                \2 可包含子模块或递归地包含子包的 Python \textcolor{red}{模块}
                \2 从技术上说,包是带有 \_\_path\_\_ 属性的 Python 模块。
                \2 简单来说,它更像是一个文件夹。
        \end{myoutline}
        \column{0.5\textwidth}
        \begin{myoutline}
            \1 标准库(Standard Library)
                \2 标准库相当于解释器的外部扩展,在我们安装 python 解释器的时候,就一块安装上了。
                \2 它并不会随着解释器的启动而启动(builtins),要想使用这些外部扩展,必须提前导入(os)。
                \2 os, sys, time等这些内置模块的的集合
            \1 第三模块,包
        %         \2 使用pip, conda等安装的网上发布的Python 扩展包或者其他语言比如 C, Rust写的扩展包
        %         \2 演示: 寻找第三方包的安装位置
            \1 sys.path
                \2 自定义模块、包——需要先放到python环境变量(sys.path)目录下后 再导入
        \end{myoutline}
    \end{columns}
\end{frame}

\begin{frame}[fragile]{演示}
    \begin{columns}
        \column{0.5\textwidth}
        \begin{myoutline}
            \1 安装一个第三方包, 如 pandas, matplotlib
                \2 导入包, 并演示功能
                \2 ctr/cmd 鼠标点击包名进入包源代码中
            \1 自己创建一个自定义模块
                \2 导入模块中的函数, 并演示功能
                \2 \_\_name\_\_变量
                    \3 运行本文件则为\_\_name\_\_
                    \3 在别的 py 文件中导入后运行则为模块名!
                \2 \_\_pycache\_\_
                    \3  python mymodule.cpython-39.pyc
        \end{myoutline}
        \column{0.5\textwidth}
        \begin{lstlisting}
def myprint():
    print("我在我的模块里!")
    print(__name__)


if __name__ == "__main__":
    myprint()
# myprint()
        \end{lstlisting}
    \end{columns}
\end{frame}


