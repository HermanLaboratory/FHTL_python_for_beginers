\subsection{项目一 序列文件的处理}

\begin{frame}[standout] 项目一 \quad 序列文件的处理 \end{frame}

\begin{frame}[standout] 序列储存格式的介绍 \end{frame}
\begin{frame}[fragile]{序列储存格式的介绍--FASTA 文件格式}
    \begin{columns}
        \column{0.5\textwidth}
        \begin{myoutline}
            \1 FASTA格式
                \2 一种用于表示核苷酸序列或多肽序列的文本格式;其中碱基对或氨基酸用单个字母来表示
                \2 允许在序列前添加序列名及注释
                \2 该格式已成为生物信息学领域的一项标准
            \1 FASTA文件各行记录信息如下:
                \2 第一行
                    \3 由大于号">"开头的任意文字说明,用于序列标记
                    \3 为了保证后续分析软件能够区分每条序列,\textcolor{red}{单个序列的标识必须是唯一的}
                \2 第二行
                    \3 序列本身; 只允许使用既定的核苷酸或氨基酸编码符号。
                    \3 通常核苷酸符号大小写均可, 而氨基酸常用大写字母。
                    \3 注意有些程序对大小写有明确要求。一般每行60-80个字母。
        \end{myoutline}
        \column{0.45\textwidth}
        \begin{lstlisting}
>HWI-D00433:463:HNT7JBCXX:1:1101:19071:2193 1:N:0:TTCTCCAT
CCTACGGGACGCATCAGTGAGGAATATTGGTCAATGGACGCGAGTCTGAACCAGCCAAGT
AGCGTGAAGGATGAAGGCCCGATGGGTTGTAAACCTCTTTTATCTGGGAATAAAACGTGC
CACGTGTGGTATTTTGTATGTACCATAAGAATAAGTATCGGCTAACTCCGTGCCAGCAGC
CGCGGTAATACGGAGGATCCGAGCGTTATCCGGATTTATTGGGTTTAAAGGGTGCGCAGG
CGGTCTGTTA
>HWI-D00433:463:HNT7JBCXX:1:1101:1713:2316 1:N:0:TTCTCCAT
CCTACGGGGTTCACCAGTAGGGAATCTTCCACAATGGGCGAAAGCCTGATTGAGCAAAGC
CGCGTGTTTTAAGAAGGTCTTCGGATCGTAAAACCCTGTTGTTAGAGAAGAAAGTGCGTG
CGCGTAACTGTTCACGTTTCTACTGTATCTAACAAGAAAGCACCGGCTAACTACGTTCCA
        \end{lstlisting}
    \end{columns}
    \footnotenoindex{https://zhuanlan.zhihu.com/p/363971040}
\end{frame}

\begin{frame}[fragile]{序列储存格式的介绍--FASTA 文件格式}
    \begin{lstlisting}
# legend server
cd /home/zhaohuanan/3.project/2022_Other_projects/2022-09-25_Prepared_data/FASTA
rsync -avzP rsync://hgdownload.cse.ucsc.edu/goldenPath/mm39/chromosomes/ .
# check一下md5
md5sum -c md5sum.txt

(base) PS C:\Users\vagrant\PythonForBioinformatics> rsync -avzP rsync://hgdownload.cse.
ucsc.edu/goldenPath/mm39/chromosomes/chrM.fa.gz .
receiving incremental file list
chrM.fa.gz
          5,385 100%    5.14MB/s    0:00:00 (xfr#1, to-chk=0/1)

sent 43 bytes  received 5,492 bytes  299.19 bytes/sec
total size is 5,385  speedup is 0.97

# 接下来我们需要将genome生成一下,未确定的基因组区域暂不考虑,只看确定的基因组
# 测试命令
echo `seq 1 1 19` X Y M | awk 'BEGIN {printf "cat "}{for(i=1; i<=NF;i++){printf "chr"$i".fa.gz "}}'
# 写入genome文件
echo `seq 1 1 19` X Y M | awk 'BEGIN {printf "cat "}{for(i=1; i<=NF;i++){printf "chr"$i".fa.gz "}}' | \
    sh > genome_ucsc_mm39.fa.gz
zcat genome_ucsc_mm39.fa.gz | grep -v N | less # linux/window
zcat < genome_ucsc_mm39.fa.gz | grep -v N | less # macos
    \end{lstlisting}
\end{frame}


\begin{frame}[fragile]{序列储存格式的介绍--FASTQ 文件格式}
    \begin{myoutline}
        \1 第一行
            \2 以@开头
            \2 后面是reads的ID以及其他信息
            \2 例如下一页例中 HWUSI-EAS100R代表Illumina设备名称,
            \2 6代表flowcell中的第六个lane,73代表第六个lane中的第73个tile
            \2 941:1973代表该read在该tile中的x:y坐标信息;
            \2 \#0, 若为多样本的混合作为输入样本,则该标志代表样本的编号,用来区分个样本中的reads/
            \2 /1代表paired end中的前一个read。
        \1 第二行
            \2 read的序列
            \2 紧接着下面两行代表该read的质量
        \1 第三行
            \2 以“+”开头,跟随着该read的名称(一般于@后面的内容相同),但有时可以省略,但“+”一定不能省
    \end{myoutline}
    \footnotenoindex{https://baike.baidu.com/item/fastQ格式 (rmspace)}
\end{frame}
\begin{frame}[fragile]{序列储存格式的介绍--FASTQ 文件格式}
    \begin{myoutline}
        \1 第四行
        \2 代表reads的质量。这一行可以详细说一下!
        \2 Illumina测序仪是按照荧光信号来判断所测序的碱基是哪一种的,例如红黄蓝绿分别对应ATCG,那么一旦出现一个紫色的信号该怎么判断呢?
        \2 因此对每个结果都有一个概率的问题。起初sanger中心用Phred quality score来衡量该read中每个碱基的质量,既-10lgP \#(lg意为log10)
            \3 其中P代表该碱基被测序错误的概率
            \3 如果该碱基测序出错的概率为0.001, 则Q应该为30,那么30+33=63,那么63对应的ASCii码为`?',则在第四行中该碱基对应的质量代表值即为`?'
            \3 ASCII https://baike.baidu.com/item/ASCII
    \end{myoutline}
    \begin{lstlisting}
@SEQ_ID
GATTTGGGGTTCAAAGCAGTATCGATCAAATAGTAAATCCATTTGTTCAACTCACAGTTT
+
!''*((((***+))%%%++)(%%%%).1***-+*''))**55CCF

# 例如在NCBI看到的FASTQ格式如下
@HWUSI-EAS100R:6:73:941:1973#0/1
GATTTGGGGTTCAAAGCAGTATCGATCAAATAGTAAATCCATTTGTTCAACTCACAGTT
+HWUSI-EAS100R:6:73:941:1973#0/1
!''*((((***+))%%%++)(%%%%).1***-+*''))**55CCF
    \end{lstlisting}
\end{frame}

\begin{frame}[fragile]{序列储存格式的介绍--FASTQ 文件格式}
    \begin{lstlisting}
# from illumina
@E00591:528:HHVW3CCX2:2:1101:17360:2170 1:N:0:GATCAGCG:ACTA
TGGAGTGAGTACGGTGTGCGTTGAAGTCCTCGTTGTCTTGTTGGCAGGGGTCTGCACCCGGGAGCCCCCGTTCTATATCATCACTGAGATCATGACCTACGGGAACCTCCTCGACTACCTCAGGGAGTGCAACTGCAACGAG
+
JJJJAFFFFJFJJJAFFJJJJFJFFJJFFFJF7A<FJ7JF-<F-7A7<JF7<FJJJ-F<JJJJFJJJJJJ<<AJFJAJJ<7AJFFFJ-<7AFFJJJA-<FF--7FJ-F7-7FA7F<FFJA-7FJJJJF<JF7F7A7-777<-
# from mgi
@Beta12AdemL1C001R00100000184/1
GTGTCCGTATCTTCACCCCACCACAAACTATTAGCTTTAGAAAGGAAAAGAAAAACCACAACAAAACAGTGTGTGTGTACCAAAGACTTATATGTGCATAAGCAAAAGCAAACAATAGCATTAGCAGGAACTCGTGGACATTCCAGGGAA
+
ICIDIIIDEDIDBIEIIIIEIIEIEEEID@DDE4IDDDEID>EIIEEEEIEEEEEIIEIEEIECEEIE4DIDIDDC6DEIIE@E6EI5CBDEDICIIECEEIIAEEEIIDEEIE:CD:IECABIIEI>E8ICI,CI9DIEDCIIE@>HEC
# from sra
@fig7_untreated_rep1.1 1 length=150
TTACAAGACTGNTGTATTAGTTTATACTACAAGGACAGGCCCATTNGNNTNTNTTNTTTTNAANTAGGGANATAGTTGGTATTAGGATTAGNATTGTTGTGAAGTATAGTANGGATGCTACTTGNCCAATGATGGTAAAAGGGTAGCTTA
+fig7_untreated_rep1.1 1 length=150
AFFFFFFFFFF#FFFFFFFFFFFFFFFFFFFFFFFFFFFFFFF/F#F##F#F#FF#FFFF#FF#FFFFFF#FFFFFFFFFFFFFFFFFFFA#FFFFFFFFFFFFFFFFFFF#FFFFFFFFFFFF#A/FFFF/FFFFFFF/FFFFFFFFFF
    \end{lstlisting}
\end{frame}