\subsection{FASTA文件的操作}
\begin{frame}[standout] FASTA文件的操作 \end{frame}
\begin{frame}{FASTA文件的操作}
    \begin{myoutline}
        \1 FASTA文件的操作
             \2 读取FASTA文件,并将其中U替换成T
             \2 读取FASTA文件,并输出反向互补序列
             \2 计算基因组序列的长度
             \2 计算基因组各染色体的平均GC含量
             \2 计算基因组中N的总长度(effective length)
    \end{myoutline}
\end{frame}

% 3. FASTA 文件的操作与计算9月24日 10:00-11:00|未开始

\begin{frame}[fragile]{Docstring-reST}
    % https://queirozf.com/entries/python-docstrings-reference-examples
    \tiny
    \begin{lstlisting}
def func(arg1, arg2):
    """Summary line.

    Extended description of function.

    :param int arg1: Description of arg1.
    :param str arg2: Description of arg2.
    :raise: ValueError if arg1 is equal to arg2
    :return: Description of return value
    :rtype: bool

    :example:

    >>> a=1
    >>> b=2
    >>> func(a,b)
    True
    """

    if arg1 == arg2:
        raise ValueError('arg1 must not be equal to arg2')

    return True
    \end{lstlisting}
\end{frame}

\begin{frame}[fragile]{Docstring-Google Style}
    \tiny
    \begin{columns}
        \column{0.6\textwidth}
        \begin{lstlisting}
def func(arg1, arg2):
    """Summary line.

    Extended description of function.

    Args:
        arg1 (int): Description of arg1
        arg2 (str): Description of arg2

    Returns:
        bool: Description of return value

    Raises:
        AttributeError: The ``Raises`` section is a list of all 
            exceptions that are relevant to the interface.
        ValueError: If `arg2` is equal to `arg1`.

    Examples:
        Examples should be written in doctest format, and should
            illustrate how to use the function.
        >>> a=1
        >>> b=2
        >>> func(a,b)
        True
    """
            \end{lstlisting}
            \column{0.4\textwidth}
            \begin{lstlisting}
    if arg1 == arg2:
        raise ValueError(
            'arg1 must not be equal to arg2'
        )

    return True
            \end{lstlisting}
    \end{columns}

\end{frame}
\begin{frame}[fragile]{Docstring-Numpy Style}
    \tiny
    \begin{columns}
        \column{0.6\textwidth}
        \begin{lstlisting}
def func(arg1, arg2):
    """Summary line.

    Extended description of function.

    Parameters
    ----------
    arg1 : int
        Description of arg1
    arg2 : str
        Description of arg2

    Returns
    -------
    bool
        Description of return value

    Raises
    ------
    AttributeError
        The ``Raises`` section is a list of all exceptions
        that are relevant to the interface.
    """
            \end{lstlisting}
            \column{0.4\textwidth}
            \begin{lstlisting}
    """
    ValueError
        If `arg2` is equal to `arg1`.

    See Also
    --------
    otherfunc: some other related function

    Examples
    --------
    These are written in doctest format, and should illustrate how to
    use the function.

    >>> a=1
    >>> b=2
    >>> func(a,b)
    True
    """

    if arg1 == arg2:
        raise ValueError(
            'arg1 must not be equal to arg2'
        )
    return True
            \end{lstlisting}
    \end{columns}

\end{frame}